\documentclass[../main.tex]{subfiles}

\usepackage{float}

\begin{document}
\begin{itemize}

    

    %observaciones
    \item El disco empezó a rebotar de izquierda a derecha, con lo que su velocidad iba disminuyendo con el tiempo.
    \item El aire dentro del disco permitió que este pueda reducir su fricción sobre la superficie donde se desplaza.
    \item Cuando el chispero electrónico este encendido y soltamos el disco desde una esquina del tablero, se empieza a formar un patrón de puntos marcados por la electricidad que en una cierta frecuencia forma una trayectoria de puntos.
    
    \item En la tabla 9 se puede observar una discrepancia en los valores de la masa, obtenidos a partir de los vectores fuerza y aceleración en cada uno de los tres puntos, lo cual contradice la segunda ley de Newton, además de una diferencia significativa en el ángulo que forman ambos vectores, especialmente en el instante t = 13 ticks donde son totalmente opuestos; asimismo, la incertidumbre final en todos los casos, supera por mucho los valores de la masa calculados, suponiendo una gran desventaja en la investigación por la poca precisión obtenida debido a la propagación de errores en los extensos cálculos realizados. 
    
    Por otra parte, el ángulo entre el vector fuerza y aceleración es variada en los tres casos, siendo el ángulo en el instante t = 7 ticks, el más cercano a 0°, que es lo teóricamente correcto según la segunda ley de Newton, pues la masa (magnitud escalar) no afecta la dirección de la aceleración, y, por tanto, el ángulo entre el vector fuerza y aceleración debería ser 0°. 
    
\end{itemize}
\end{document}