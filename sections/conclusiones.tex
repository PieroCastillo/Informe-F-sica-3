\documentclass[../main.tex]{subfiles}
\begin{document}
Cuando la fuerza de cada resorte empieza a disminuir con los cambios de trayectoria del centro del disco cada vez que transcurre un tick, empieza a describir un movimiento amortiguado vibratorio. Lo que quiere decir, que el trabajo va disminuyendo mientras la aceleración disminuye con menos distancia recorrida entre ticks consecutivos.\\
La trayectoria del centro del disco tiene la forma de un espiral, lo que significa que como el disco tiene adentro aire, esto permite que, con las fuerzas de elasticidad de los resortes y el aire, el disco se pueda mover de manera oscilante. \\
La gran discrepancia encontrada, se presume que tiene origen en errores sistemáticos, siendo más notorio en la calibración de los resortes, ya que, al medir la constante K de los resortes y en la obtención de resultados se observa que tiene un margen de error considerable, en el caso del resorte B. Además, la perdida de precisión presentada a su vez yace en las aproximaciones consideradas en la guía, tales como la aproximación de la velocidad instantánea con la velocidad media entre dos puntos contiguos y asociar a ese trayecto de los mismos puntos, una fuerza que permanece constante hasta que pase uno de los puntos. \\

\end{document}