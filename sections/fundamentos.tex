\documentclass[../main.tex]{subfiles}

\begin{document}
\subsection{Sistema de Referencia Inercial}
Es un sistema de referencia el cuál presenta \textit{velocidad constante}, o lo que es lo mismo, \textit{aceleración nula}.
\subsection{Cantidad de Movimiento o Momentum \( ( \overrightarrow{p} ) \)}
La \textbf{cantidad de movimiento} se define como \textbf{el producto de la masa por su velocidad}.\cite{finn}
\[ \vec{p} = m \cdot \vec{v} \]
Es una cantidad vectorial y tiene la msima dirección que la velocidad. Su importancia radica en que combina los 2 elementosque caracterizan el estado dinámico de una partícula: su masa y su velocidad.\cite{finn}
\subsection{Conservación de la Cantidad de Movimiento}
Es un principio que dice \textit{la cantidad de movimiento total de un sistema aislado de fuerzas externas permanece constante}.
Se puede escribir de la siguiente manera:\cite{finn}
\[ \vec{p_{0}} = \vec{p_{1}} \]
Donde:
\begin{itemize}
    \item \(\vec{p_{0}}\): Cantidad de movimiento inicial del sistema.
    \item \(\vec{p_{1}}\): Cantidad de movimiento final del sistema.
\end{itemize}
\subsection{Fuerza  \( ( \vec{F} ) \)}
Es un concepto matemático que nos permite entender la causa del movimiento de una partícula. Se define como \textbf{la derivada de la cantidad de movimiento con respecto al tiempo}.\cite{finn}
\[ \vec{F} = \frac{d \vec{p}}{dt} \]
Usando definición de la cantidad de movimiento:
\[ \vec{F} = \frac{d  (m \cdot \vec{v}) }{dt} \]
Si la masa es constante, entonces:
\[ \vec{F} = m \cdot \frac{d \vec{v} }{dt} \]
Pero la derivada de la velocidad con respecto al tiempo es la aceleración, entonces queda:
\[ \vec{F}  = m \cdot \vec{a} \]

\subsection{Segunda Ley de Newton}
Esta ley dicta lo siquiente: \\ \\
\textit{Si una fuerza externa neta actúa sobre un cuerpo, éste se acelera. La dirección de la aceleración es la misma 
que la de la fuerza neta. El vector de fuerza neta es igual a la masa del cuerpo 
multiplicada por su aceleración.} \cite{sears} \\ \\
Se escribe de la siguiente manera:
\[ \sum \vec{F}  = m \cdot \vec{a_r} \]
Se debe dar la aclaración de que esta ley solo es válida en un \textit{sistema de referencia inercial}.
\end{document}